\chapter{A Estrutura}
\label{cap:estrutura}

\section{Definição}
\label{sec:definicao}
uma árvore binária de busca (ou árvore binária de pesquisa) é uma estrutura de dados de árvore binária baseada em nós, onde todos os nós da subárvore esquerda possuem um valor numérico inferior ao nó raiz e todos os nós da subárvore direita possuem um valor superior ao nó raiz.
\cite{ABBwiki}

\section{Elementos e Terminologias}
\label{sec:elementos}
\begin{itemize}
    \item Nós - são todos os itens guardados na árvore
    \item Raiz - é o nó do topo da árvore 
    \item Filhos - são os nós que vem depois dos outros nós 
    \item Pais - são os nós que vem antes dos outros nós 
    \item Folhas - são os nós que não têm filhos; são os últimos nós da árvore
    \item Grau - número de sub-árvores relacionadas àquele nó.  
    \item Altura - comprimento do caminho mais longo do nó a uma folha 
    \item Profundidade - comprimento do unico caminho entre o nó raiz e o nó avaliado
\end{itemize}